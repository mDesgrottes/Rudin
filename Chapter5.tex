\documentclass{article}
\usepackage{amsmath}
\renewcommand{\baselinestretch}{1.05}
\newcommand{\norm}[1]{\left\lVert#1\right\rVert}
\usepackage{amsmath,amsthm,verbatim,amssymb,amsfonts,amscd, graphicx}
\usepackage[T1]{fontenc}
\usepackage{bigfoot} % to allow verbatim in footnote
\usepackage[numbered,framed]{matlab-prettifier}
\usepackage{filecontents}
\usepackage{graphics}
\topmargin0.0cm
\headheight0.0cm
\headsep0.0cm
\oddsidemargin0.0cm
\textheight23.0cm
\textwidth16.5cm
\footskip1.0cm
\theoremstyle{plain}
\newtheorem{theorem}{Theorem}
\newtheorem{corollary}{Corollary}
\newtheorem{lemma}{Lemma}
\newtheorem{proposition}{Proposition}
\newtheorem*{surfacecor}{Corollary 1}
\newtheorem{conjecture}{Conjecture} 
\newtheorem{question}{Question} 
\theoremstyle{definition}
\newtheorem{definition}{Definition}
\title{Analysis 1}
\author{Mike Desgrottes}
\date{October 2020}

\begin{document}
\maketitle

\begin{question}
	Suppose $f^{'}(x) > 0$ in $(a,b)$. Prove that f is strictly increasing in $(a,b)$, and let g be its inverse function. Prove that g is differentiable, and that $$g^{'}(f(x)) = \frac{1}{f^{'}(x)}   (a < x < b) $$.

\end{question}

\begin{proof}
	Let $x,y \in (a,b)$ with $y - x > 0$. By the mean value theorem, there exists a $c \in (x,y)$ such that $$f^{'}(c) = \frac{f(y) - f(x)}{y - x} > 0 \implies f(y) - f(x) > 0$$. 

	$g(f(x)) = x$ which is differentiable. Its derivative is $g^{'}(f(x))f^{'}(x) = 1$ by the chain rule. Hence $g^{'}(f(x)) = \frac{1}{f^{'}(x)}$ . g is differentiable on $(a,b)$ because $f^{'}(x) > 0$.
\end{proof}

\begin{question}
	Suppose 

	(a) f is continuous on $[0,\infty)$

	(b) f is differentiable on $(0, \infty)$

	(c) f(0) = 0

	(d) $f^{'}$ is monotonically increasing.

	Prove that $$ g(x) = \frac{f(x)}{x}  (x > 0)$$ is monotonically increasing.
\end{question}

\begin{proof}
	This is a consequence of the Mean Value Theorem which guarantee the existence of $c >0$ such that $f^{'}(c) = \frac{f(x) - f(y)}{x - y}$ with $y = 0$. We get $g(x) = f^{'}(c)$ which is monotonically increasing.
\end{proof}

\begin{question}
	Let f be a continuous real function on R, of which it is known that $f^{'}(x)$ exists for all $x \not = 0$ and that $f^{'}(x) \to 3$ as $x \to 0$. Does it follow that $f^{'}(0)$ exists?.
\end{question}

\begin{proof}
	Yes, it exists. To see this, let $\{ y_{n} \}$ be a sequence that converges to 0. The Mean Value Theorem guarantee the existence of a sequence of $c_{n}$ such that $f^{'}(c_{n}) = \frac{f(y_{n}) - f(0)}{y_{n}}$ where $c_{n} \to 0$. Thus, $f^{'}(0) = \lim_{n \to \infty} f^{'}(c_{n}) = \lim_{n \to \infty} \frac{f(y_{n}) - f(0)}{y_{n}}$.

	$$\lim_{y_{n} \to 0} \frac{f^{'}(y_{n})}{1} = 3 $$.

	By L'Hospital rule, and continuity of f, $f^{'}(0) = 3$.
\end{proof}

\begin{question}
	Suppose $a \in \mathbb{R}$, f is twice differentiable real function on $(a,\infty)$, and $M_{0},M_{1},M_{2}$ are the least upper bounds of $|f(x)|, |f^{'}(x)|, |f^{''}(x)|$ respectively on $(a,\infty)$. Prove that $M_{1}^{2} \leq 4M_{0}M_{2}$
\end{question}

\begin{proof}
	By Taylor's theorem, $$ f(x) = P(x) + \frac{f^{(n)}(c)}{n!}(x - \alpha)^{n}$$. With $n = 2, \alpha = x + 2h$, We have $$ f(x) = f(x + 2h) + f^{'}(x + 2h)(-2h) + f^{''}(c)(2h^{2}) $$. We will show that $$f^{''}(c) = \frac{f^{'}(x + 2h) - f^{'}(x)}{2h}$$ Set $g(t) = f(t) - f(\alpha) - f^{'}(\alpha)(t - \alpha) - \frac{f^{''}(c)}{4}$ with second derivative $$ g^{'}(t) = f^{'}(t) - f^{'}(\alpha) - \frac{f^{''}(c)(t - \alpha)}{2}$$. Set $t = x, \alpha = x + 2h$, we get the result.

	$$ f(x) = f(x + 2h) - 4h^{2}f^{''}(c) - 2hf^{'}(x) + 2h^{2}f^{''}(c)$$.

	$$f^{'}(x) = \frac{f(x + 2h) - f(x) - 2h^{2}f^{''}(c)}{2h} $$.

	$$ |f^{'}(x)| \leq M_{1}\leq \frac{M_{0} + h^{2}M_{2}}{h} \implies M_{1}^{2} \leq 1 + 3M_{0}M_{1} \leq 4M_{0}M_{1}$$ when $h = M_{0}$.
\end{proof}

\begin{question}
	Suppose f is a real, three times differentiable function on $[-1,1]$, such that $$f(-1) = 0, f(0) = 0, f(1) = 1, f^{'}(0) = 0 $$.

	Prove that $f^{'''}(x) \geq 3$ for some $x \in (-1, 1)$.
\end{question}

\begin{proof}
	Taylor's theorem gives us $$f(1) = f(0) + f^{'}(0) + \frac{f^{''}(0)}{2} + \frac{f^{'''}(c)}6{} $$ and $$f(-1) = f(0) - f^{'}(0) + \frac{f^{''}(0)}{2} - \frac{f^{'''}(s)}{6} $$ and $$f^{'''}(c) = f^{''}(1) - f^{''}(0) $$ and $$f^{'''}(s) = f^{''}(-1) - f^{''}(0) $$ which gives us $$ f^{'''}(s) + f^{'''}(c) = 6$$ with $c \in (0,1)$ and $(-1, 0)$. One of the two is at least 3. The result follows.
\end{proof}
\end{document}


\documentclass{article}
\usepackage{amsmath}
\renewcommand{\baselinestretch}{1.05}
\newcommand{\norm}[1]{\left\lVert#1\right\rVert}
\usepackage{amsmath,amsthm,verbatim,amssymb,amsfonts,amscd, graphicx}
\usepackage[T1]{fontenc}
\usepackage{bigfoot} % to allow verbatim in footnote
\usepackage[numbered,framed]{matlab-prettifier}
\usepackage{filecontents}
\usepackage{graphics}
\topmargin0.0cm
\headheight0.0cm
\headsep0.0cm
\oddsidemargin0.0cm
\textheight23.0cm
\textwidth16.5cm
\footskip1.0cm
\theoremstyle{plain}
\newtheorem{theorem}{Theorem}
\newtheorem{corollary}{Corollary}
\newtheorem{lemma}{Lemma}
\newtheorem{proposition}{Proposition}
\newtheorem*{surfacecor}{Corollary 1}
\newtheorem{conjecture}{Conjecture} 
\newtheorem{question}{Question} 
\theoremstyle{definition}
\newtheorem{definition}{Definition}
\title{Chapter 4}
\author{Mike Desgrottes}
\date{October 2020}

\begin{document}
\maketitle

\begin{theorem}
	Suppose $f$ is a real function defined on $\mathbb{R}$ which statisfies $$\lim_{h \to 0} [f(x + h) - f(x - h)]= 0 $$ for every real number x. Does this imply that f is continuous?
\end{theorem}

\begin{proof}
	No, the function $f(x) = x^2$ if $x \not = 0$ and $f(0) = 1$. This function is discontinuous at 0.
\end{proof}
\end{document}

